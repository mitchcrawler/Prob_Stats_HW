
% Default to the notebook output style

    


% Inherit from the specified cell style.




    
\documentclass[11pt]{article}

    
    
    \usepackage[T1]{fontenc}
    % Nicer default font (+ math font) than Computer Modern for most use cases
    \usepackage{mathpazo}

    % Basic figure setup, for now with no caption control since it's done
    % automatically by Pandoc (which extracts ![](path) syntax from Markdown).
    \usepackage{graphicx}
    % We will generate all images so they have a width \maxwidth. This means
    % that they will get their normal width if they fit onto the page, but
    % are scaled down if they would overflow the margins.
    \makeatletter
    \def\maxwidth{\ifdim\Gin@nat@width>\linewidth\linewidth
    \else\Gin@nat@width\fi}
    \makeatother
    \let\Oldincludegraphics\includegraphics
    % Set max figure width to be 80% of text width, for now hardcoded.
    \renewcommand{\includegraphics}[1]{\Oldincludegraphics[width=.8\maxwidth]{#1}}
    % Ensure that by default, figures have no caption (until we provide a
    % proper Figure object with a Caption API and a way to capture that
    % in the conversion process - todo).
    \usepackage{caption}
    \DeclareCaptionLabelFormat{nolabel}{}
    \captionsetup{labelformat=nolabel}

    \usepackage{adjustbox} % Used to constrain images to a maximum size 
    \usepackage{xcolor} % Allow colors to be defined
    \usepackage{enumerate} % Needed for markdown enumerations to work
    \usepackage{geometry} % Used to adjust the document margins
    \usepackage{amsmath} % Equations
    \usepackage{amssymb} % Equations
    \usepackage{textcomp} % defines textquotesingle
    % Hack from http://tex.stackexchange.com/a/47451/13684:
    \AtBeginDocument{%
        \def\PYZsq{\textquotesingle}% Upright quotes in Pygmentized code
    }
    \usepackage{upquote} % Upright quotes for verbatim code
    \usepackage{eurosym} % defines \euro
    \usepackage[mathletters]{ucs} % Extended unicode (utf-8) support
    \usepackage[utf8x]{inputenc} % Allow utf-8 characters in the tex document
    \usepackage{fancyvrb} % verbatim replacement that allows latex
    \usepackage{grffile} % extends the file name processing of package graphics 
                         % to support a larger range 
    % The hyperref package gives us a pdf with properly built
    % internal navigation ('pdf bookmarks' for the table of contents,
    % internal cross-reference links, web links for URLs, etc.)
    \usepackage{hyperref}
    \usepackage{longtable} % longtable support required by pandoc >1.10
    \usepackage{booktabs}  % table support for pandoc > 1.12.2
    \usepackage[inline]{enumitem} % IRkernel/repr support (it uses the enumerate* environment)
    \usepackage[normalem]{ulem} % ulem is needed to support strikethroughs (\sout)
                                % normalem makes italics be italics, not underlines
    

    
    
    % Colors for the hyperref package
    \definecolor{urlcolor}{rgb}{0,.145,.698}
    \definecolor{linkcolor}{rgb}{.71,0.21,0.01}
    \definecolor{citecolor}{rgb}{.12,.54,.11}

    % ANSI colors
    \definecolor{ansi-black}{HTML}{3E424D}
    \definecolor{ansi-black-intense}{HTML}{282C36}
    \definecolor{ansi-red}{HTML}{E75C58}
    \definecolor{ansi-red-intense}{HTML}{B22B31}
    \definecolor{ansi-green}{HTML}{00A250}
    \definecolor{ansi-green-intense}{HTML}{007427}
    \definecolor{ansi-yellow}{HTML}{DDB62B}
    \definecolor{ansi-yellow-intense}{HTML}{B27D12}
    \definecolor{ansi-blue}{HTML}{208FFB}
    \definecolor{ansi-blue-intense}{HTML}{0065CA}
    \definecolor{ansi-magenta}{HTML}{D160C4}
    \definecolor{ansi-magenta-intense}{HTML}{A03196}
    \definecolor{ansi-cyan}{HTML}{60C6C8}
    \definecolor{ansi-cyan-intense}{HTML}{258F8F}
    \definecolor{ansi-white}{HTML}{C5C1B4}
    \definecolor{ansi-white-intense}{HTML}{A1A6B2}

    % commands and environments needed by pandoc snippets
    % extracted from the output of `pandoc -s`
    \providecommand{\tightlist}{%
      \setlength{\itemsep}{0pt}\setlength{\parskip}{0pt}}
    \DefineVerbatimEnvironment{Highlighting}{Verbatim}{commandchars=\\\{\}}
    % Add ',fontsize=\small' for more characters per line
    \newenvironment{Shaded}{}{}
    \newcommand{\KeywordTok}[1]{\textcolor[rgb]{0.00,0.44,0.13}{\textbf{{#1}}}}
    \newcommand{\DataTypeTok}[1]{\textcolor[rgb]{0.56,0.13,0.00}{{#1}}}
    \newcommand{\DecValTok}[1]{\textcolor[rgb]{0.25,0.63,0.44}{{#1}}}
    \newcommand{\BaseNTok}[1]{\textcolor[rgb]{0.25,0.63,0.44}{{#1}}}
    \newcommand{\FloatTok}[1]{\textcolor[rgb]{0.25,0.63,0.44}{{#1}}}
    \newcommand{\CharTok}[1]{\textcolor[rgb]{0.25,0.44,0.63}{{#1}}}
    \newcommand{\StringTok}[1]{\textcolor[rgb]{0.25,0.44,0.63}{{#1}}}
    \newcommand{\CommentTok}[1]{\textcolor[rgb]{0.38,0.63,0.69}{\textit{{#1}}}}
    \newcommand{\OtherTok}[1]{\textcolor[rgb]{0.00,0.44,0.13}{{#1}}}
    \newcommand{\AlertTok}[1]{\textcolor[rgb]{1.00,0.00,0.00}{\textbf{{#1}}}}
    \newcommand{\FunctionTok}[1]{\textcolor[rgb]{0.02,0.16,0.49}{{#1}}}
    \newcommand{\RegionMarkerTok}[1]{{#1}}
    \newcommand{\ErrorTok}[1]{\textcolor[rgb]{1.00,0.00,0.00}{\textbf{{#1}}}}
    \newcommand{\NormalTok}[1]{{#1}}
    
    % Additional commands for more recent versions of Pandoc
    \newcommand{\ConstantTok}[1]{\textcolor[rgb]{0.53,0.00,0.00}{{#1}}}
    \newcommand{\SpecialCharTok}[1]{\textcolor[rgb]{0.25,0.44,0.63}{{#1}}}
    \newcommand{\VerbatimStringTok}[1]{\textcolor[rgb]{0.25,0.44,0.63}{{#1}}}
    \newcommand{\SpecialStringTok}[1]{\textcolor[rgb]{0.73,0.40,0.53}{{#1}}}
    \newcommand{\ImportTok}[1]{{#1}}
    \newcommand{\DocumentationTok}[1]{\textcolor[rgb]{0.73,0.13,0.13}{\textit{{#1}}}}
    \newcommand{\AnnotationTok}[1]{\textcolor[rgb]{0.38,0.63,0.69}{\textbf{\textit{{#1}}}}}
    \newcommand{\CommentVarTok}[1]{\textcolor[rgb]{0.38,0.63,0.69}{\textbf{\textit{{#1}}}}}
    \newcommand{\VariableTok}[1]{\textcolor[rgb]{0.10,0.09,0.49}{{#1}}}
    \newcommand{\ControlFlowTok}[1]{\textcolor[rgb]{0.00,0.44,0.13}{\textbf{{#1}}}}
    \newcommand{\OperatorTok}[1]{\textcolor[rgb]{0.40,0.40,0.40}{{#1}}}
    \newcommand{\BuiltInTok}[1]{{#1}}
    \newcommand{\ExtensionTok}[1]{{#1}}
    \newcommand{\PreprocessorTok}[1]{\textcolor[rgb]{0.74,0.48,0.00}{{#1}}}
    \newcommand{\AttributeTok}[1]{\textcolor[rgb]{0.49,0.56,0.16}{{#1}}}
    \newcommand{\InformationTok}[1]{\textcolor[rgb]{0.38,0.63,0.69}{\textbf{\textit{{#1}}}}}
    \newcommand{\WarningTok}[1]{\textcolor[rgb]{0.38,0.63,0.69}{\textbf{\textit{{#1}}}}}
    
    
    % Define a nice break command that doesn't care if a line doesn't already
    % exist.
    \def\br{\hspace*{\fill} \\* }
    % Math Jax compatability definitions
    \def\gt{>}
    \def\lt{<}
    % Document parameters
    \title{Final Project MCOT Outreaches to Crisis Line Calls}
    
    
    

    % Pygments definitions
    
\makeatletter
\def\PY@reset{\let\PY@it=\relax \let\PY@bf=\relax%
    \let\PY@ul=\relax \let\PY@tc=\relax%
    \let\PY@bc=\relax \let\PY@ff=\relax}
\def\PY@tok#1{\csname PY@tok@#1\endcsname}
\def\PY@toks#1+{\ifx\relax#1\empty\else%
    \PY@tok{#1}\expandafter\PY@toks\fi}
\def\PY@do#1{\PY@bc{\PY@tc{\PY@ul{%
    \PY@it{\PY@bf{\PY@ff{#1}}}}}}}
\def\PY#1#2{\PY@reset\PY@toks#1+\relax+\PY@do{#2}}

\expandafter\def\csname PY@tok@w\endcsname{\def\PY@tc##1{\textcolor[rgb]{0.73,0.73,0.73}{##1}}}
\expandafter\def\csname PY@tok@c\endcsname{\let\PY@it=\textit\def\PY@tc##1{\textcolor[rgb]{0.25,0.50,0.50}{##1}}}
\expandafter\def\csname PY@tok@cp\endcsname{\def\PY@tc##1{\textcolor[rgb]{0.74,0.48,0.00}{##1}}}
\expandafter\def\csname PY@tok@k\endcsname{\let\PY@bf=\textbf\def\PY@tc##1{\textcolor[rgb]{0.00,0.50,0.00}{##1}}}
\expandafter\def\csname PY@tok@kp\endcsname{\def\PY@tc##1{\textcolor[rgb]{0.00,0.50,0.00}{##1}}}
\expandafter\def\csname PY@tok@kt\endcsname{\def\PY@tc##1{\textcolor[rgb]{0.69,0.00,0.25}{##1}}}
\expandafter\def\csname PY@tok@o\endcsname{\def\PY@tc##1{\textcolor[rgb]{0.40,0.40,0.40}{##1}}}
\expandafter\def\csname PY@tok@ow\endcsname{\let\PY@bf=\textbf\def\PY@tc##1{\textcolor[rgb]{0.67,0.13,1.00}{##1}}}
\expandafter\def\csname PY@tok@nb\endcsname{\def\PY@tc##1{\textcolor[rgb]{0.00,0.50,0.00}{##1}}}
\expandafter\def\csname PY@tok@nf\endcsname{\def\PY@tc##1{\textcolor[rgb]{0.00,0.00,1.00}{##1}}}
\expandafter\def\csname PY@tok@nc\endcsname{\let\PY@bf=\textbf\def\PY@tc##1{\textcolor[rgb]{0.00,0.00,1.00}{##1}}}
\expandafter\def\csname PY@tok@nn\endcsname{\let\PY@bf=\textbf\def\PY@tc##1{\textcolor[rgb]{0.00,0.00,1.00}{##1}}}
\expandafter\def\csname PY@tok@ne\endcsname{\let\PY@bf=\textbf\def\PY@tc##1{\textcolor[rgb]{0.82,0.25,0.23}{##1}}}
\expandafter\def\csname PY@tok@nv\endcsname{\def\PY@tc##1{\textcolor[rgb]{0.10,0.09,0.49}{##1}}}
\expandafter\def\csname PY@tok@no\endcsname{\def\PY@tc##1{\textcolor[rgb]{0.53,0.00,0.00}{##1}}}
\expandafter\def\csname PY@tok@nl\endcsname{\def\PY@tc##1{\textcolor[rgb]{0.63,0.63,0.00}{##1}}}
\expandafter\def\csname PY@tok@ni\endcsname{\let\PY@bf=\textbf\def\PY@tc##1{\textcolor[rgb]{0.60,0.60,0.60}{##1}}}
\expandafter\def\csname PY@tok@na\endcsname{\def\PY@tc##1{\textcolor[rgb]{0.49,0.56,0.16}{##1}}}
\expandafter\def\csname PY@tok@nt\endcsname{\let\PY@bf=\textbf\def\PY@tc##1{\textcolor[rgb]{0.00,0.50,0.00}{##1}}}
\expandafter\def\csname PY@tok@nd\endcsname{\def\PY@tc##1{\textcolor[rgb]{0.67,0.13,1.00}{##1}}}
\expandafter\def\csname PY@tok@s\endcsname{\def\PY@tc##1{\textcolor[rgb]{0.73,0.13,0.13}{##1}}}
\expandafter\def\csname PY@tok@sd\endcsname{\let\PY@it=\textit\def\PY@tc##1{\textcolor[rgb]{0.73,0.13,0.13}{##1}}}
\expandafter\def\csname PY@tok@si\endcsname{\let\PY@bf=\textbf\def\PY@tc##1{\textcolor[rgb]{0.73,0.40,0.53}{##1}}}
\expandafter\def\csname PY@tok@se\endcsname{\let\PY@bf=\textbf\def\PY@tc##1{\textcolor[rgb]{0.73,0.40,0.13}{##1}}}
\expandafter\def\csname PY@tok@sr\endcsname{\def\PY@tc##1{\textcolor[rgb]{0.73,0.40,0.53}{##1}}}
\expandafter\def\csname PY@tok@ss\endcsname{\def\PY@tc##1{\textcolor[rgb]{0.10,0.09,0.49}{##1}}}
\expandafter\def\csname PY@tok@sx\endcsname{\def\PY@tc##1{\textcolor[rgb]{0.00,0.50,0.00}{##1}}}
\expandafter\def\csname PY@tok@m\endcsname{\def\PY@tc##1{\textcolor[rgb]{0.40,0.40,0.40}{##1}}}
\expandafter\def\csname PY@tok@gh\endcsname{\let\PY@bf=\textbf\def\PY@tc##1{\textcolor[rgb]{0.00,0.00,0.50}{##1}}}
\expandafter\def\csname PY@tok@gu\endcsname{\let\PY@bf=\textbf\def\PY@tc##1{\textcolor[rgb]{0.50,0.00,0.50}{##1}}}
\expandafter\def\csname PY@tok@gd\endcsname{\def\PY@tc##1{\textcolor[rgb]{0.63,0.00,0.00}{##1}}}
\expandafter\def\csname PY@tok@gi\endcsname{\def\PY@tc##1{\textcolor[rgb]{0.00,0.63,0.00}{##1}}}
\expandafter\def\csname PY@tok@gr\endcsname{\def\PY@tc##1{\textcolor[rgb]{1.00,0.00,0.00}{##1}}}
\expandafter\def\csname PY@tok@ge\endcsname{\let\PY@it=\textit}
\expandafter\def\csname PY@tok@gs\endcsname{\let\PY@bf=\textbf}
\expandafter\def\csname PY@tok@gp\endcsname{\let\PY@bf=\textbf\def\PY@tc##1{\textcolor[rgb]{0.00,0.00,0.50}{##1}}}
\expandafter\def\csname PY@tok@go\endcsname{\def\PY@tc##1{\textcolor[rgb]{0.53,0.53,0.53}{##1}}}
\expandafter\def\csname PY@tok@gt\endcsname{\def\PY@tc##1{\textcolor[rgb]{0.00,0.27,0.87}{##1}}}
\expandafter\def\csname PY@tok@err\endcsname{\def\PY@bc##1{\setlength{\fboxsep}{0pt}\fcolorbox[rgb]{1.00,0.00,0.00}{1,1,1}{\strut ##1}}}
\expandafter\def\csname PY@tok@kc\endcsname{\let\PY@bf=\textbf\def\PY@tc##1{\textcolor[rgb]{0.00,0.50,0.00}{##1}}}
\expandafter\def\csname PY@tok@kd\endcsname{\let\PY@bf=\textbf\def\PY@tc##1{\textcolor[rgb]{0.00,0.50,0.00}{##1}}}
\expandafter\def\csname PY@tok@kn\endcsname{\let\PY@bf=\textbf\def\PY@tc##1{\textcolor[rgb]{0.00,0.50,0.00}{##1}}}
\expandafter\def\csname PY@tok@kr\endcsname{\let\PY@bf=\textbf\def\PY@tc##1{\textcolor[rgb]{0.00,0.50,0.00}{##1}}}
\expandafter\def\csname PY@tok@bp\endcsname{\def\PY@tc##1{\textcolor[rgb]{0.00,0.50,0.00}{##1}}}
\expandafter\def\csname PY@tok@fm\endcsname{\def\PY@tc##1{\textcolor[rgb]{0.00,0.00,1.00}{##1}}}
\expandafter\def\csname PY@tok@vc\endcsname{\def\PY@tc##1{\textcolor[rgb]{0.10,0.09,0.49}{##1}}}
\expandafter\def\csname PY@tok@vg\endcsname{\def\PY@tc##1{\textcolor[rgb]{0.10,0.09,0.49}{##1}}}
\expandafter\def\csname PY@tok@vi\endcsname{\def\PY@tc##1{\textcolor[rgb]{0.10,0.09,0.49}{##1}}}
\expandafter\def\csname PY@tok@vm\endcsname{\def\PY@tc##1{\textcolor[rgb]{0.10,0.09,0.49}{##1}}}
\expandafter\def\csname PY@tok@sa\endcsname{\def\PY@tc##1{\textcolor[rgb]{0.73,0.13,0.13}{##1}}}
\expandafter\def\csname PY@tok@sb\endcsname{\def\PY@tc##1{\textcolor[rgb]{0.73,0.13,0.13}{##1}}}
\expandafter\def\csname PY@tok@sc\endcsname{\def\PY@tc##1{\textcolor[rgb]{0.73,0.13,0.13}{##1}}}
\expandafter\def\csname PY@tok@dl\endcsname{\def\PY@tc##1{\textcolor[rgb]{0.73,0.13,0.13}{##1}}}
\expandafter\def\csname PY@tok@s2\endcsname{\def\PY@tc##1{\textcolor[rgb]{0.73,0.13,0.13}{##1}}}
\expandafter\def\csname PY@tok@sh\endcsname{\def\PY@tc##1{\textcolor[rgb]{0.73,0.13,0.13}{##1}}}
\expandafter\def\csname PY@tok@s1\endcsname{\def\PY@tc##1{\textcolor[rgb]{0.73,0.13,0.13}{##1}}}
\expandafter\def\csname PY@tok@mb\endcsname{\def\PY@tc##1{\textcolor[rgb]{0.40,0.40,0.40}{##1}}}
\expandafter\def\csname PY@tok@mf\endcsname{\def\PY@tc##1{\textcolor[rgb]{0.40,0.40,0.40}{##1}}}
\expandafter\def\csname PY@tok@mh\endcsname{\def\PY@tc##1{\textcolor[rgb]{0.40,0.40,0.40}{##1}}}
\expandafter\def\csname PY@tok@mi\endcsname{\def\PY@tc##1{\textcolor[rgb]{0.40,0.40,0.40}{##1}}}
\expandafter\def\csname PY@tok@il\endcsname{\def\PY@tc##1{\textcolor[rgb]{0.40,0.40,0.40}{##1}}}
\expandafter\def\csname PY@tok@mo\endcsname{\def\PY@tc##1{\textcolor[rgb]{0.40,0.40,0.40}{##1}}}
\expandafter\def\csname PY@tok@ch\endcsname{\let\PY@it=\textit\def\PY@tc##1{\textcolor[rgb]{0.25,0.50,0.50}{##1}}}
\expandafter\def\csname PY@tok@cm\endcsname{\let\PY@it=\textit\def\PY@tc##1{\textcolor[rgb]{0.25,0.50,0.50}{##1}}}
\expandafter\def\csname PY@tok@cpf\endcsname{\let\PY@it=\textit\def\PY@tc##1{\textcolor[rgb]{0.25,0.50,0.50}{##1}}}
\expandafter\def\csname PY@tok@c1\endcsname{\let\PY@it=\textit\def\PY@tc##1{\textcolor[rgb]{0.25,0.50,0.50}{##1}}}
\expandafter\def\csname PY@tok@cs\endcsname{\let\PY@it=\textit\def\PY@tc##1{\textcolor[rgb]{0.25,0.50,0.50}{##1}}}

\def\PYZbs{\char`\\}
\def\PYZus{\char`\_}
\def\PYZob{\char`\{}
\def\PYZcb{\char`\}}
\def\PYZca{\char`\^}
\def\PYZam{\char`\&}
\def\PYZlt{\char`\<}
\def\PYZgt{\char`\>}
\def\PYZsh{\char`\#}
\def\PYZpc{\char`\%}
\def\PYZdl{\char`\$}
\def\PYZhy{\char`\-}
\def\PYZsq{\char`\'}
\def\PYZdq{\char`\"}
\def\PYZti{\char`\~}
% for compatibility with earlier versions
\def\PYZat{@}
\def\PYZlb{[}
\def\PYZrb{]}
\makeatother


    % Exact colors from NB
    \definecolor{incolor}{rgb}{0.0, 0.0, 0.5}
    \definecolor{outcolor}{rgb}{0.545, 0.0, 0.0}



    
    % Prevent overflowing lines due to hard-to-break entities
    \sloppy 
    % Setup hyperref package
    \hypersetup{
      breaklinks=true,  % so long urls are correctly broken across lines
      colorlinks=true,
      urlcolor=urlcolor,
      linkcolor=linkcolor,
      citecolor=citecolor,
      }
    % Slightly bigger margins than the latex defaults
    
    \geometry{verbose,tmargin=1in,bmargin=1in,lmargin=1in,rmargin=1in}
    
    

    \begin{document}
    
    
    \maketitle
    
    

    
    Final Project

    Is There a Correlation between Crisis Line Call Volume and MCOT
Outreaches?

    Abstract (No more than 250 words)

    Are Crisis Line call volumes at the University of Utah Neuropsychiatric
Institute's (UNI) Crisis Line correlated with the number of outreaches
the UNI Mobile Crisis Outreach Team (MCOT) respond to? This is the
question I wanted to know the answer to and the topic of this project.
MCOT responds to calls anywhere in Salt Lake County from multiple
requesting sources: police, family, friends, co-workers, other
healthcare/mental health workers, and service providers in the
community. The Crisis Line receives calls from the whole state of Utah.

    Introduction: Data Hypothesis

    The idea of having a mobile crisis outreach team is not a new one; many
cities, counties, and states have been adding them in recent years.
However, an actual count of teams in the United States was not found,
but performing a ``Google'' search for the following: ``mobile crisis
outreach teams'', results in links to pages regarding MCOT teams from
all over the country. In addition to not being able to locate a count of
the number teams, there did not appear to be any available information
regarding the statistics of if the number of outreaches are correlated
to the number of crisis calls received within a certain jurisdiction.
That being the case, there was no prior data to compare the results in
this project to. The hypothesis tested, using the ``R'' programming
language, was: ``Is there a statistical significant level of correlation
between the number of crisis calls received by the UNI Crisis Line and
the number of outreaches that the UNI MCOT team responds to?'' The null
hypothesis, thus being: ``There is no significant correlation between
the number of crisis calls received by the UNI Crisis Line and the
number of outreaches that the UNI MCOT team responds to.'' The data used
in this project was obtained from the web-based software that both the
UNI Crisis Line and UNI MCOT use to record call information in,
https://www.icarol.com. The data contains number of calls and number of
outreaches from March 1, 2012 (the month the UNI Mobile Crisis Outreach
Team first started) to March 31, 2018. In total the data covers 2,222
days, the sum of Crisis Line calls for this period totals 241,354, and
the sum for number of MCOT outreaches totals 21,442. There were zero
days where the Crisis Line took 0 calls and 49 days that MCOT had zero
outreaches(nearly all of these days occurred within the first two months
of the data, as the program was just launched in the month of March
2012). The maximum number of calls for the Crisis Line was 191 calls,
which occurred on March 12, 2018, and the maximum number of outreaches
for MCOT was 29, which occurred on September 22, 2016; the mean number
of calls for Crisis Line was 108.6 and the mean for MCOT was 9.65. The
minimum number of calls for Crisis line was 59 and MCOT was 0.

    Methods: Descriptive Statistics Tests

    \begin{Verbatim}[commandchars=\\\{\}]
{\color{incolor}In [{\color{incolor}20}]:} \PY{c+c1}{\PYZsh{}install and load necessary packages}
         \PY{k+kn}{library}\PY{p}{(}gtools\PY{p}{)}
         install.packages\PY{p}{(}\PY{l+s}{\PYZdq{}}\PY{l+s}{mosaic\PYZdq{}}\PY{p}{)}
         \PY{k+kn}{library}\PY{p}{(}mosaic\PY{p}{)}
         \PY{k+kn}{require}\PY{p}{(}ggplot2\PY{p}{)}
         \PY{k+kn}{require}\PY{p}{(}sandwich\PY{p}{)}
         \PY{k+kn}{require}\PY{p}{(}msm\PY{p}{)}
\end{Verbatim}


    \begin{Verbatim}[commandchars=\\\{\}]
Updating HTML index of packages in '.Library'
Making 'packages.html' {\ldots} done

    \end{Verbatim}

    \begin{Verbatim}[commandchars=\\\{\}]
{\color{incolor}In [{\color{incolor}2}]:} \PY{c+c1}{\PYZsh{}read in my data files}
        MCOT \PY{o}{=} read.csv\PY{p}{(}file \PY{o}{=} \PY{l+s}{\PYZdq{}}\PY{l+s}{MCOT outreach data from March 1 2012 to March 31 2018.csv\PYZdq{}}\PY{p}{,} header \PY{o}{=} \PY{n+nb+bp}{F}\PY{p}{)}
        Crisis \PY{o}{=} read.csv\PY{p}{(}file \PY{o}{=} \PY{l+s}{\PYZdq{}}\PY{l+s}{Crisis Line data from March 1 2012 to March 31 2018.csv\PYZdq{}}\PY{p}{,} header \PY{o}{=} \PY{n+nb+bp}{F}\PY{p}{)}
        \PY{c+c1}{\PYZsh{}give the columns names}
        \PY{k+kp}{colnames}\PY{p}{(}MCOT\PY{p}{)} \PY{o}{\PYZlt{}\PYZhy{}} \PY{k+kt}{c}\PY{p}{(}\PY{l+s}{\PYZdq{}}\PY{l+s}{Date\PYZdq{}}\PY{p}{,} \PY{l+s}{\PYZdq{}}\PY{l+s}{MCOT\PYZus{}Calls\PYZdq{}}\PY{p}{)}
        \PY{k+kp}{colnames}\PY{p}{(}Crisis\PY{p}{)} \PY{o}{\PYZlt{}\PYZhy{}} \PY{k+kt}{c}\PY{p}{(}\PY{l+s}{\PYZdq{}}\PY{l+s}{Date\PYZdq{}}\PY{p}{,} \PY{l+s}{\PYZdq{}}\PY{l+s}{Crisis\PYZus{}Calls\PYZdq{}}\PY{p}{)}
        \PY{c+c1}{\PYZsh{}subset the Crisis line file to get rid of the column that is in both MCOT and Crisis: \PYZdq{}Date\PYZdq{}}
        Crisis\PYZus{}1 \PY{o}{\PYZlt{}\PYZhy{}} \PY{k+kp}{subset}\PY{p}{(}Crisis\PY{p}{,} select \PY{o}{=} \PY{o}{\PYZhy{}}\PY{k+kt}{c}\PY{p}{(}Date\PY{p}{)}\PY{p}{)}
        \PY{c+c1}{\PYZsh{}combine the two files}
        MCOTandCrisis \PY{o}{=} \PY{k+kp}{cbind}\PY{p}{(}MCOT\PY{p}{,} Crisis\PYZus{}1 \PY{p}{)}
\end{Verbatim}


    \begin{Verbatim}[commandchars=\\\{\}]
{\color{incolor}In [{\color{incolor}3}]:} \PY{c+c1}{\PYZsh{}take an initial look at the data by plotting the two variables}
        matplot\PY{p}{(}y \PY{o}{=} MCOTandCrisis\PY{p}{,} type \PY{o}{=} \PY{l+s}{\PYZsq{}}\PY{l+s}{l\PYZsq{}}\PY{p}{,} lty \PY{o}{=} \PY{l+m}{1}\PY{p}{)}
\end{Verbatim}


    \begin{Verbatim}[commandchars=\\\{\}]
Warning message in xy.coords(x, y, xlabel, ylabel, log = log):
“NAs introduced by coercion”Warning message in xy.coords(x, y, xlabel, ylabel, log):
“NAs introduced by coercion”
    \end{Verbatim}

    \begin{center}
    \adjustimage{max size={0.9\linewidth}{0.9\paperheight}}{output_9_1.png}
    \end{center}
    { \hspace*{\fill} \\}
    
    \begin{Verbatim}[commandchars=\\\{\}]
{\color{incolor}In [{\color{incolor}4}]:} \PY{c+c1}{\PYZsh{}get the total number of days covered in the data}
        \PY{k+kp}{length}\PY{p}{(}MCOTandCrisis\PY{o}{\PYZdl{}}Date\PY{p}{)}
\end{Verbatim}


    2222

    
    \begin{Verbatim}[commandchars=\\\{\}]
{\color{incolor}In [{\color{incolor}5}]:} \PY{c+c1}{\PYZsh{}get the total sum of all Crisis Calls and MCOT outreaches}
        \PY{k+kp}{sum}\PY{p}{(}MCOTandCrisis\PY{o}{\PYZdl{}}Crisis\PYZus{}Calls\PY{p}{)}
        \PY{k+kp}{sum}\PY{p}{(}MCOTandCrisis\PY{o}{\PYZdl{}}MCOT\PYZus{}Calls\PY{p}{)}
\end{Verbatim}


    241354

    
    21442

    
    \begin{Verbatim}[commandchars=\\\{\}]
{\color{incolor}In [{\color{incolor}6}]:} \PY{c+c1}{\PYZsh{}count of how many days where Crisis Line and MCOT took 0 calls/0 outreaches}
        count\PY{p}{(}MCOTandCrisis\PY{o}{\PYZdl{}}Crisis\PYZus{}Calls \PY{o}{==} \PY{l+m}{0}\PY{p}{)}
        count\PY{p}{(}MCOTandCrisis\PY{o}{\PYZdl{}}MCOT\PYZus{}Calls \PY{o}{==} \PY{l+m}{0}\PY{p}{)}
\end{Verbatim}


    \textbf{TRUE:} 0

    
    \textbf{TRUE:} 49

    
    \begin{Verbatim}[commandchars=\\\{\}]
{\color{incolor}In [{\color{incolor}7}]:} \PY{c+c1}{\PYZsh{}find which day the max values occurred for Crisis and MCOT}
        \PY{k+kp}{which.max}\PY{p}{(}MCOTandCrisis\PY{o}{\PYZdl{}}Crisis\PYZus{}Calls\PY{p}{)}
        \PY{k+kp}{which.max}\PY{p}{(}MCOTandCrisis\PY{o}{\PYZdl{}}MCOT\PYZus{}Calls\PY{p}{)}
\end{Verbatim}


    2203

    
    1667

    
    \begin{Verbatim}[commandchars=\\\{\}]
{\color{incolor}In [{\color{incolor}8}]:} \PY{c+c1}{\PYZsh{}Show the results for the corresponding index for max value for both Crisis and MCOT}
        \PY{k+kp}{print}\PY{p}{(}MCOTandCrisis\PY{p}{[}\PY{l+m}{2203}\PY{p}{,}\PY{p}{]}\PY{p}{)}
        \PY{k+kp}{print}\PY{p}{(}MCOTandCrisis\PY{p}{[}\PY{l+m}{1667}\PY{p}{,}\PY{p}{]}\PY{p}{)}
\end{Verbatim}


    \begin{Verbatim}[commandchars=\\\{\}]
          Date MCOT\_Calls Crisis\_Calls
2203 3/12/2018         17          191
          Date MCOT\_Calls Crisis\_Calls
1667 9/22/2016         29          118

    \end{Verbatim}

    \begin{Verbatim}[commandchars=\\\{\}]
{\color{incolor}In [{\color{incolor}9}]:} \PY{c+c1}{\PYZsh{}summarize the combined data}
        \PY{k+kp}{summary}\PY{p}{(}MCOTandCrisis\PY{p}{)}
\end{Verbatim}


    
    \begin{verbatim}
       Date        MCOT_Calls     Crisis_Calls  
 1/1/2013:   1   Min.   : 0.00   Min.   : 59.0  
 1/1/2014:   1   1st Qu.: 6.00   1st Qu.: 94.0  
 1/1/2015:   1   Median : 9.00   Median :108.5  
 1/1/2016:   1   Mean   : 9.65   Mean   :108.6  
 1/1/2017:   1   3rd Qu.:13.00   3rd Qu.:122.0  
 1/1/2018:   1   Max.   :29.00   Max.   :191.0  
 (Other) :2216                                  
    \end{verbatim}

    
    \begin{Verbatim}[commandchars=\\\{\}]
{\color{incolor}In [{\color{incolor}10}]:} \PY{c+c1}{\PYZsh{}perform a poisson regression (because the data is on counts) on the two variables, perform a summary of the}
         \PY{c+c1}{\PYZsh{}poisson regression, and put it into a new object to be used later}
         \PY{k+kp}{summary}\PY{p}{(}m1 \PY{o}{\PYZlt{}\PYZhy{}} glm\PY{p}{(}MCOT\PYZus{}Calls \PY{o}{\PYZti{}} Crisis\PYZus{}Calls\PY{p}{,} data \PY{o}{=} MCOTandCrisis\PY{p}{,} family\PY{o}{=}\PY{l+s}{\PYZdq{}}\PY{l+s}{poisson\PYZdq{}}\PY{p}{)}\PY{p}{)}
\end{Verbatim}


    
    \begin{verbatim}

Call:
glm(formula = MCOT_Calls ~ Crisis_Calls, family = "poisson", 
    data = MCOTandCrisis)

Deviance Residuals: 
    Min       1Q   Median       3Q      Max  
-4.5849  -1.1836  -0.1501   0.8955   5.1889  

Coefficients:
              Estimate Std. Error z value Pr(>|z|)    
(Intercept)  0.8819926  0.0385443   22.88   <2e-16 ***
Crisis_Calls 0.0124610  0.0003336   37.35   <2e-16 ***
---
Signif. codes:  0 ‘***’ 0.001 ‘**’ 0.01 ‘*’ 0.05 ‘.’ 0.1 ‘ ’ 1

(Dispersion parameter for poisson family taken to be 1)

    Null deviance: 6675.5  on 2221  degrees of freedom
Residual deviance: 5303.9  on 2220  degrees of freedom
AIC: 13987

Number of Fisher Scoring iterations: 5

    \end{verbatim}

    
    \begin{Verbatim}[commandchars=\\\{\}]
{\color{incolor}In [{\color{incolor}11}]:} \PY{c+c1}{\PYZsh{}Calculate the robust standard errors and calculate the p\PYZhy{}value, also calculated the 95\PYZpc{} confidence interval}
         
         \PY{c+c1}{\PYZsh{}perform a goodness of fit test with the residual deviance to see if the data fits the model, also calculate}
         \PY{c+c1}{\PYZsh{}the lower and upper limits with a 95\PYZpc{} confidence interval}
         cov.m1 \PY{o}{\PYZlt{}\PYZhy{}} vcovHC\PY{p}{(}m1\PY{p}{,} type\PY{o}{=}\PY{l+s}{\PYZdq{}}\PY{l+s}{HC0\PYZdq{}}\PY{p}{)}
         std.err \PY{o}{\PYZlt{}\PYZhy{}} \PY{k+kp}{sqrt}\PY{p}{(}\PY{k+kp}{diag}\PY{p}{(}cov.m1\PY{p}{)}\PY{p}{)}
         r.est \PY{o}{\PYZlt{}\PYZhy{}} \PY{k+kp}{cbind}\PY{p}{(}Estimate\PY{o}{=} coef\PY{p}{(}m1\PY{p}{)}\PY{p}{,} \PY{l+s}{\PYZdq{}}\PY{l+s}{Robust SE\PYZdq{}} \PY{o}{=} std.err\PY{p}{,}
         \PY{l+s}{\PYZdq{}}\PY{l+s}{Pr(\PYZgt{}|z|)\PYZdq{}} \PY{o}{=} \PY{l+m}{2} \PY{o}{*} pnorm\PY{p}{(}\PY{k+kp}{abs}\PY{p}{(}coef\PY{p}{(}m1\PY{p}{)}\PY{o}{/}std.err\PY{p}{)}\PY{p}{,} lower.tail\PY{o}{=}\PY{k+kc}{FALSE}\PY{p}{)}\PY{p}{,}
         LL \PY{o}{=} coef\PY{p}{(}m1\PY{p}{)} \PY{o}{\PYZhy{}} \PY{l+m}{1.96} \PY{o}{*} std.err\PY{p}{,}
         UL \PY{o}{=} coef\PY{p}{(}m1\PY{p}{)} \PY{o}{+} \PY{l+m}{1.96} \PY{o}{*} std.err\PY{p}{)}
         
         r.est
\end{Verbatim}


    \begin{tabular}{r|lllll}
  & Estimate & Robust SE & Pr(>\textbar{}z\textbar{}) & LL & UL\\
\hline
	(Intercept) & 0.88199261    & 0.055091917   &  1.097338e-57 & 0.77401245    & 0.98997277   \\
	Crisis\_Calls & 0.01246105    & 0.000471533   & 6.766013e-154 & 0.01153684    & 0.01338525   \\
\end{tabular}


    
    \begin{Verbatim}[commandchars=\\\{\}]
{\color{incolor}In [{\color{incolor}12}]:} \PY{c+c1}{\PYZsh{}perform a goodness of fit test with the residual deviance to see if the data fits the model, also calculate}
         \PY{c+c1}{\PYZsh{}the lower and upper limits with a 95\PYZpc{} confidence interval}
         \PY{k+kp}{with}\PY{p}{(}m1\PY{p}{,} \PY{k+kp}{cbind}\PY{p}{(}res.deviance \PY{o}{=} deviance\PY{p}{,} df \PY{o}{=} df.residual\PY{p}{,}
           p \PY{o}{=} pchisq\PY{p}{(}deviance\PY{p}{,} df.residual\PY{p}{,} lower.tail\PY{o}{=}\PY{k+kc}{FALSE}\PY{p}{)}\PY{p}{)}\PY{p}{)}
\end{Verbatim}


    \begin{tabular}{lll}
 res.deviance & df & p\\
\hline
	 5303.882      & 2220          & 1.336443e-252\\
\end{tabular}


    
    \begin{Verbatim}[commandchars=\\\{\}]
{\color{incolor}In [{\color{incolor}13}]:} \PY{c+c1}{\PYZsh{}The calculated p value was highly significant, thus indicating that the model was not a good fit.}
         \PY{c+c1}{\PYZsh{}conduct an Analysis of Variance test on the data}
         fit \PY{o}{\PYZlt{}\PYZhy{}} aov\PY{p}{(}MCOT\PYZus{}Calls \PY{o}{\PYZti{}} Crisis\PYZus{}Calls\PY{p}{,} data\PY{o}{=}MCOTandCrisis\PY{p}{)}
         fit
\end{Verbatim}


    
    \begin{verbatim}
Call:
   aov(formula = MCOT_Calls ~ Crisis_Calls, data = MCOTandCrisis)

Terms:
                Crisis_Calls Residuals
Sum of Squares      13476.94  46722.65
Deg. of Freedom            1      2220

Residual standard error: 4.587618
Estimated effects may be unbalanced
    \end{verbatim}

    
    \begin{Verbatim}[commandchars=\\\{\}]
{\color{incolor}In [{\color{incolor}14}]:} \PY{c+c1}{\PYZsh{}plot the aov to determine correct model}
         layout\PY{p}{(}\PY{k+kt}{matrix}\PY{p}{(}\PY{k+kt}{c}\PY{p}{(}\PY{l+m}{1}\PY{p}{,}\PY{l+m}{2}\PY{p}{,}\PY{l+m}{3}\PY{p}{,}\PY{l+m}{4}\PY{p}{)}\PY{p}{,}\PY{l+m}{2}\PY{p}{,}\PY{l+m}{2}\PY{p}{)}\PY{p}{)} \PY{c+c1}{\PYZsh{} optional layout }
         plot\PY{p}{(}fit\PY{p}{)} \PY{c+c1}{\PYZsh{} diagnostic plots}
\end{Verbatim}


    \begin{center}
    \adjustimage{max size={0.9\linewidth}{0.9\paperheight}}{output_20_0.png}
    \end{center}
    { \hspace*{\fill} \\}
    
    \begin{Verbatim}[commandchars=\\\{\}]
{\color{incolor}In [{\color{incolor}15}]:} \PY{c+c1}{\PYZsh{}The data fits a normal q\PYZhy{}q plot best.  Plot the data as a normal q\PYZhy{}q plot.}
         MCOTandCrisis \PY{o}{\PYZpc{}\PYZgt{}\PYZpc{}}
         \PY{k+kp}{sample}\PY{p}{(}\PY{l+m}{500}\PY{p}{)} \PY{o}{\PYZpc{}\PYZgt{}\PYZpc{}}
         ggplot\PY{p}{(}aes\PY{p}{(}x \PY{o}{=} Crisis\PYZus{}Calls\PY{p}{,} y \PY{o}{=} MCOT\PYZus{}Calls\PY{p}{)}\PY{p}{)} \PY{o}{+}
         geom\PYZus{}point\PY{p}{(}\PY{p}{)} \PY{o}{+} 
         stat\PYZus{}smooth\PY{p}{(}method \PY{o}{=} lm\PY{p}{,} se \PY{o}{=} \PY{l+m}{0}\PY{p}{)} \PY{o}{+} 
         stat\PYZus{}smooth\PY{p}{(}method \PY{o}{=} loess\PY{p}{,} se \PY{o}{=} \PY{l+m}{0}\PY{p}{,} color \PY{o}{=} \PY{l+s}{\PYZdq{}}\PY{l+s}{green\PYZdq{}}\PY{p}{)} \PY{o}{+} 
         xlab\PY{p}{(}\PY{l+s}{\PYZdq{}}\PY{l+s}{Crisis Calls\PYZdq{}}\PY{p}{)} \PY{o}{+} ylab\PY{p}{(}\PY{l+s}{\PYZdq{}}\PY{l+s}{MCOT calls\PYZdq{}}\PY{p}{)}
\end{Verbatim}


    
    
    \begin{center}
    \adjustimage{max size={0.9\linewidth}{0.9\paperheight}}{output_21_1.png}
    \end{center}
    { \hspace*{\fill} \\}
    
    \begin{Verbatim}[commandchars=\\\{\}]
{\color{incolor}In [{\color{incolor}16}]:} \PY{c+c1}{\PYZsh{}perform a correlation test using the pearson method which measures linear correlation between two variables}
         \PY{c+c1}{\PYZsh{}the resulting values are between \PYZhy{}1, 0, and 1; where \PYZhy{}1 means there is total negative linear correlation,}
         \PY{c+c1}{\PYZsh{}0 means that ther is no correlation, and 1 means there is total positve linear correlation.(1) }
         cor\PYZus{}res \PY{o}{\PYZlt{}\PYZhy{}} cor.test\PY{p}{(}MCOTandCrisis\PY{o}{\PYZdl{}}MCOT\PYZus{}Calls\PY{p}{,} MCOTandCrisis\PY{o}{\PYZdl{}}Crisis\PYZus{}Calls\PY{p}{,} 
                         method \PY{o}{=} \PY{l+s}{\PYZdq{}}\PY{l+s}{pearson\PYZdq{}}\PY{p}{)}
         cor\PYZus{}res
\end{Verbatim}


    
    \begin{verbatim}

	Pearson's product-moment correlation

data:  x and y
t = 25.305, df = 2220, p-value < 2.2e-16
alternative hypothesis: true correlation is not equal to 0
95 percent confidence interval:
 0.4402283 0.5048013
sample estimates:
      cor 
0.4731501 

    \end{verbatim}

    
    \begin{Verbatim}[commandchars=\\\{\}]
{\color{incolor}In [{\color{incolor}17}]:} \PY{c+c1}{\PYZsh{}create new variable for the linear regression of the data and calculate it\PYZsq{}s coefficients}
         mod \PY{o}{\PYZlt{}\PYZhy{}} lm\PY{p}{(}MCOT\PYZus{}Calls\PY{o}{\PYZti{}}Crisis\PYZus{}Calls\PY{p}{,} data \PY{o}{=} MCOTandCrisis\PY{p}{)} 
         coef\PY{p}{(}mod\PY{p}{)}
\end{Verbatim}


    \begin{description*}
\item[(Intercept)] -3.76553092634256
\item[Crisis\textbackslash{}\_Calls] 0.123507419468222
\end{description*}


    
    \begin{Verbatim}[commandchars=\\\{\}]
{\color{incolor}In [{\color{incolor}18}]:} \PY{c+c1}{\PYZsh{}calculate the r squared of the new variable for linear regression of the data.}
         \PY{c+c1}{\PYZsh{}\PYZdq{}R\PYZhy{}squared is the fraction by which the variance of the errors is less than the variance of the dependent variable.\PYZdq{}}
         \PY{c+c1}{\PYZsh{}\PYZhy{}(2)}
         rsquared\PY{p}{(}mod\PY{p}{)}
\end{Verbatim}


    0.223870985132211

    
    \begin{Verbatim}[commandchars=\\\{\}]
{\color{incolor}In [{\color{incolor}19}]:} \PY{c+c1}{\PYZsh{}provide a summary of the mod variable for linear regression of the data}
         \PY{k+kp}{summary}\PY{p}{(}mod\PY{p}{)}
\end{Verbatim}


    
    \begin{verbatim}

Call:
lm(formula = MCOT_Calls ~ Crisis_Calls, data = MCOTandCrisis)

Residuals:
     Min       1Q   Median       3Q      Max 
-11.2666  -3.3472  -0.4259   2.8092  18.1917 

Coefficients:
              Estimate Std. Error t value Pr(>|t|)    
(Intercept)  -3.765531   0.539005  -6.986 3.72e-12 ***
Crisis_Calls  0.123507   0.004881  25.305  < 2e-16 ***
---
Signif. codes:  0 ‘***’ 0.001 ‘**’ 0.01 ‘*’ 0.05 ‘.’ 0.1 ‘ ’ 1

Residual standard error: 4.588 on 2220 degrees of freedom
Multiple R-squared:  0.2239,	Adjusted R-squared:  0.2235 
F-statistic: 640.3 on 1 and 2220 DF,  p-value: < 2.2e-16

    \end{verbatim}

    
    Results - Conclusion

    The hypothesis that there is a correlation between the number of calls
taken by the UNI Crisis Line and the number of outreaches performed by
UNI MCOT was correct, meaning that there was statistically significant
correlation between these two variables. The Pearson's correlation was
highly significant and indicated a correlation of approximately 0.4732,
or nearly a 50\% positive correlation for number of outreaches to number
of crisis calls. The Pearson correlation on the variables showed a 95\%
confidence interval for correlation to be between 0.4402 and 0.5048. The
R-squared value, which is used to determine what percent of the variance
is explained by the model, was 0.2239. The adjusted R-squared value,
which looks at the standard error of the regression, was 0.2235.
Although there is correlation between the two variables, the variance
only accounts for approsimately 22\% of the correlation in the model.
This data suggests that there are likely other factors at play that
contribute to the number of outreaches than just the call volume of the
Crisis Line and would require further analysis.

    What's next

    Because Crisis Line call volumes only partially explain the correlation
of number of outreaches for the UNI MCOT, more analysis looking at other
variables would be required in order to better explain what else
contributes to the number of outreaches performed. A few of the
variables that come to mind that may play a role are: staffing levels of
both MCOT and the Crisis Line, staffing levels of the various police
departments in the county of Salt Lake, police call volumes, hospital
bed availability in the community, seasonal patterns, and weather
conditions to name a few.

    References

    \begin{enumerate}
\def\labelenumi{(\arabic{enumi})}
\item
  Pearson correlation coefficient {[}Internet{]}. Wikipedia. Wikimedia
  Foundation; 2018 {[}cited 2018Apr29{]}. Available from:
  https://en.wikipedia.org/wiki/Pearson\_correlation\_coefficient
\item
  Fuqua School of Business. {[}Internet{]}. What's a good value for
  R-squared? {[}cited 2018Apr29{]}. Available from:
  https://people.duke.edu/\textasciitilde{}rnau/rsquared.htm
\end{enumerate}


    % Add a bibliography block to the postdoc
    
    
    
    \end{document}
